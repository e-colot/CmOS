\documentclass[a4paper,12pt]{article}

% Language and encoding
\usepackage[utf8]{inputenc}
\usepackage[T1]{fontenc}
\usepackage[english]{babel}

% Comments block
\usepackage{verbatim}

% Mathematics and symbols
\usepackage{amsmath, amssymb, amsfonts}

% Figures and images
\usepackage{graphicx}
\usepackage{float}
\usepackage{caption}
\usepackage{subcaption}

% Tables
\usepackage{array}
\usepackage{booktabs}
\usepackage{multirow}
\usepackage{xcolor}

% Code listings
\usepackage{listings}
\usepackage{courier}
\lstset{
    basicstyle=\ttfamily\footnotesize,
    keywordstyle=\color{blue}\bfseries,
    commentstyle=\color{gray},
    stringstyle=\color{red},
    breaklines=true
}

% Hyperlinks
\usepackage{hyperref}
\hypersetup{
    colorlinks=true,
    linkcolor=blue,
    citecolor=red,
    urlcolor=cyan
}

% Thick lines in tables
\makeatletter
\newcommand{\thickhline}{%
    \noalign {\ifnum 0=`}\fi \hrule height 2pt
    \futurelet \reserved@a \@xhline
}
\makeatother

% Page layout
\usepackage{geometry}
\geometry{a4paper, margin=1in}

% Title information
\title{CmOS}
\date{2025}
\begin{document}

\maketitle


\section{Introduction}

In this file, the main concepts used to build CmOS (\textit{C modeled OS}) will be detailed. The objective of this report is to complete the content of the comments that can be found in the code.\\
The goal of this project is to understand and apply the main principles of an OS without having to struggle with time demanding parts such as assembly code, bootloader, drivers and so on. Because it is a simulation of an OS done in C, it suits better to the definition of an hypervisor with a single virtual machine than to the one of an operating system.

\section{Storage}

\subsection{Generalities}

The storage of CmOS is a file located in \texttt{bin/disk} with a parametric size. To interface with it, \texttt{src/disk.c} gives functions to initialize, write and read it.

\subsection{File system}

The files will be split in pages of 16 bytes. It is a small amount but as the system built is quite small, it is enough. \\
Because the whole storage is 4kB, there is room for 256 pages. The first 2 pages will be used to store a bitmap (needing 256 bits which corresponds to 32 bytes).\\
This will be followed by a file allocation table (FAT). Because of the unrealistically small size of the storage, the programs will not be named but will be attributed with a 1 byte identifier. The next byte will be a pointer to the page at which the file starts. In addition to this, each page that compose the FAT will start with twice 0xFF to differentiate it from a program page. This means that the FAT will take 7 programs per used page.\\
Each page will start with a 1 byte pointer to the next page. This pointer will be set to 0x00 for the last page of the file.

\section{Programs}

\subsection{Generalities}

For a classic OS, this step would not have been necessary because the programs are written in machine code and that is what is then run by the computer. However, the choice of building a dedicated simple low-level language has been made such that small programs can run on CmOS.\\
To begin with it, each instruction will consists of an instruction code followed by up to 3 arguments. The RAM size that was set at the beginning was 256 bytes, which leads to an 8-bit architecture so any pointer can be stored.
%In a way to make it more flexible, it will be able to run in either 8-bit or 16-bit configuration. this means that an instruction can consist of 4 or 8 bytes.

\subsection{CPU}

The CPU will have 16 bytes of registers. This way, a register can be accessed with a 4-bit identifiers. This allows any instruction to only take a maximum of 16 bits of arguments. These 16 bits can be up to 4 registers, 2 registers and an immediate byte value or an immediate 2 bytes value (disk address for example).\\
RL and RH appears in a non natural order because the 16-bit registers are stored in little-endian (LSB before MSB).

\begin{table}[H]
    \centering
    \begin{tabular}{|c|c|c|c|}
        \hline
        register name & address  & size & use \\ 
        \hline
        FLAGS  & 0x0 & 8 bits  & to be determined \\
        R1     & 0x1 & 8 bits  & general purpose register \\
        R2     & 0x2 & 8 bits  & general purpose register \\
        R3     & 0x3 & 8 bits  & general purpose register \\
        R4     & 0x4 & 8 bits  & general purpose register \\
        R5     & 0x5 & 8 bits  & general purpose register \\
        R16    & 0x6 & 16 bits & 16 bit register for arithmetic \\
        RL     & 0x6 & 8 bits  & general purpose register \\
        RH     & 0x7 & 8 bits  & general purpose register \\
        RSI    & 0x8 & 16 bits & source pointer \\
        RDI    & 0xA & 16 bits & destination pointer \\
        RI     & 0xC & 16 bits & instruction pointer \\
        RS     & 0xE & 16 bits & stack pointer \\
        \hline
    \end{tabular}
    \caption{CPU registers}
    \label{tab:CPU registers}
\end{table}

\begin{table}[h]
    \centering
    \begin{tabular}{|c|c|c|c|c|c|c|c|}
        \hline
        CF  & ZF  & SF  & OF & $\star$ & $\star$ & $\star$ & $\star$ \\
        \hline
        Carry Flag & Zero Flag & Sign Flag & Overflow Flag & & & & \\
        \hline
    \end{tabular}
    \caption{FLAGS register}
    \label{tab:flags}
\end{table}

A main difference between this simulation and a real OS is that here the OS will not run on the CPU. This means that there is no need for any interruptions as CmOS has full control on the CPU. This is because it is from the OS code that the CPU will be triggered.

\subsection{Instruction set}
\subsubsection{8-bit configuration}

In the instruction table, the following convention is followed:
\begin{itemize}
    \item r* refers to a register (4 bits)
    \item imm* refers to an immediate value
    \item $\star$ indicates an unused field
    \item an \textit{italic} field indicates an input
    \item a \textbf{bold} field indicates an output
\end{itemize}

\begin{table}[H]
    \centering
    \begin{tabular}{|c|c|c|c|c|r|}
        \hline
        \multirow{2}{*}{Instruction} & \multirow{2}{*}{Op Code}  & \multirow{2}{*}{byte1} & \multirow{2}{*}{byte2} & 16 bit reg & \multicolumn{1}{c|}{\multirow{2}{*}{comment}}\\
         & & & & compatibility & \multicolumn{1}{c|}{} \\
        \thickhline
        % 0x0- for bitwise operations
        \multirow{2}{*}{AND} & 0x00 & \textit{r1} \quad \textit{r2}  & $\star$ \quad \textbf{r3} & & \textbf{r3} = \textit{r1} AND \textit{r2}\\
        \cline{2-6}
                             & 0x01 & \textit{r1} \quad \textbf{r2}  & \textit{imm8} & & \textbf{r2} = \textit{r1} AND \textit{imm8}\\
        \hline
        \multirow{2}{*}{OR} & 0x02 & \textit{r1} \quad \textit{r2}  & $\star$ \quad \textbf{r3} & & \textbf{r3} = \textit{r1} OR \textit{r2}\\
        \cline{2-6}
                            & 0x03 & \textit{r1} \quad \textbf{r2}  & \textit{imm8} & & \textbf{r2} = \textit{r1} OR \textit{imm8}\\
        \hline
        NOT & 0x04 & \textit{r1} \quad \textbf{r2} & $\star$ & & \textbf{r2} = NOT \textit{r1} \\
        \hline
        \multirow{2}{*}{SHL} & 0x05 & \textit{r1} \quad \textit{r2} & $\star$ \quad \textbf{r3} & $\bullet$ & \textbf{r3} = \textit{r1} $\ll$ \textit{r2} \\
        \cline{2-6}
                            & 0x06 & \textit{r1} \quad \textbf{r2} & \textit{imm8} & $\bullet$ & \textbf{r2} = \textit{r1} $\ll$ \textit{imm8} \\
        \hline
        \multirow{2}{*}{SHR} & 0x07 & \textit{r1} \quad \textit{r2} & $\star$ \quad \textbf{r3} & $\bullet$ & \textbf{r3} = \textit{r1} $\gg$ \textit{r2} \\
        \cline{2-6}
                            & 0x08 & \textit{r1} \quad \textbf{r2} & \textit{imm8} & $\bullet$ & \textbf{r2} = \textit{r1} $\gg$ \textit{imm8} \\
        \thickhline
        % 0x1- for arithmetic operations
        \multirow{2}{*}{ADD} & 0x10 & \textit{r1} \quad \textit{r2}  & $\star$ \quad \textbf{r3} & $\bullet$ & \textbf{r3} = \textit{r1} + \textit{r2}\\
        \cline{2-6}
                             & 0x11 & \textit{r1} \quad \textbf{r2}  & \textit{imm8} & $\bullet$ & \textbf{r2} = \textit{r1} + \textit{imm8}\\
        \hline
        \multirow{3}{*}{SUB} & 0x12 & \textit{r1} \quad \textit{r2}  & $\star$ \quad \textbf{r3} & $\bullet$ & \textbf{r3} = \textit{r1} - \textit{r2}\\
        \cline{2-6}
                             & 0x13 & \textit{r1} \quad \textbf{r2}  & \textit{imm8} & $\bullet$ & \textbf{r2} = \textit{r1} - \textit{imm8}\\
        \cline{2-6}
                             & 0x14 & \textit{imm8} &  \textit{r1} \quad \textbf{r2} & & \textbf{r2} = \textit{imm8} - \textit{r1}\\
        \hline
        \multirow{2}{*}{MUL} & 0x15 & \textit{r1} \quad \textit{r2}  & $\star$ & & \textbf{R16} = \textit{r1} * \textit{r2}\\
        \cline{2-6}
                             & 0x16 & $\star$ \quad \textit{r1}  & \textit{imm8} & & \textbf{R16} = \textit{r1} * \textit{imm8}\\
        \hline
        \multirow{2}{*}{IDIV}& 0x17 & \textit{r1} \quad \textbf{r2}  & $\star$ & & \textbf{r2} = \textit{R16} // \textit{r1}\\
        \cline{2-6}
                             & 0x18 & \textit{imm8} & $\star$ \quad \textbf{r1} & & \textbf{r1} = \textit{R16} // \textit{imm8}\\
        \hline
        \multirow{2}{*}{MOD} & 0x19 & \textit{r1} \quad \textbf{r2}  & $\star$ & & \textbf{r2} = \textit{R16} mod \textit{r1}\\
        \cline{2-6}
                             & 0x1A & \textit{imm8} & $\star$ \quad \textbf{r1} & & \textbf{r1} = \textit{R16} mod \textit{imm8}\\
        \thickhline
        % 0x2- for storage and memory access
        \multirow{3}{*}{MOV} & 0x20 & \textit{r1} \quad \textbf{r2} & $\star$ & $\bullet$ & \textbf{r2} = \textit{r1}\\
        \cline{2-6}
                             & 0x21 & $\star$ \quad \textbf{r1} & \textit{imm8} & & \textbf{r1} = \textit{imm8}\\
        \cline{2-6}
                             & 0x22 & \multicolumn{2}{c|}{\textit{imm16}} & $\bullet$ & \textbf{R16} = \textit{imm16} \\
        \hline
        \multirow{2}{*}{LOAD} & 0x23 & $\star$ \quad \textbf{r1} & $\star$ & & \textbf{r1} = *(\textit{RSI})\\
        \cline{2-6}
                             & 0x24 & $\star$ & $\star$ & & \textbf{R16} = *(\textit{RSI})\\
        \hline
        \multirow{2}{*}{STORE} & 0x25 & $\star$ \quad \textit{r1} & $\star$ & & *(\textbf{RDI}) = \textit{r1}\\
        \cline{2-6}
                             & 0x26 & $\star$ & $\star$ & & *(\textbf{RDI}) = \textit{R16}\\
        \hline
        REGDUMP & 0x27 & $\star$ & $\star$ & & stores all the registers in \textbf{RDI}\\
        \hline
        REGFILL & 0x28 & $\star$ & $\star$ & & loads \textit{RSI} in all the registers\\
        \thickhline
        % 0x3- for flags only operations
        CMP & 0x30 & \textit{r1} \quad \textit{r2} & $\star$ & $\bullet$ & raises \textbf{ZF}, \textbf{SF} for (\textit{r2} - \textit{r1})\\
        \hline
        \multirow{4}{*}{TEST} & \multirow{2}{*}{0x31} & \multirow{2}{*}{\textit{r1} \quad \textit{r2}} & \multirow{2}{*}{$\star$} & \multirow{2}{*}{$\bullet$} & raises \textbf{ZF} for \\
         & & & & & \textit{r1} AND 0b1 $\ll$ \textit{r2}\\
        \cline{2-6}
                             & \multirow{2}{*}{0x32} & \multirow{2}{*}{$\star$ \quad \textit{r1}} & \multirow{2}{*}{\textit{imm8}} & \multirow{2}{*}{$\bullet$} & raises \textbf{ZF} for \\
                             & & & & & \textit{r1} AND 0b1 $\ll$ \textit{imm8}\\
        \thickhline
        % 0x4- for conditions
        \multirow{2}{*}{SKIFZ} & \multirow{2}{*}{0x40} & \multirow{2}{*}{$\star$} & \multirow{2}{*}{$\star$} & & skip next instruction\\
         & & & & & if \textit{ZF} = 1\\
        \cline{2-6}
        \multirow{2}{*}{SKIFNZ} & \multirow{2}{*}{0x41} & \multirow{2}{*}{$\star$} & \multirow{2}{*}{$\star$} & & skip next instruction\\
         & & & & & if \textit{ZF} = 0\\
        \cline{2-6}
        \thickhline

        % 0xF- for sys calls
        PRNT & 0xF0 & $\star$ \quad \textit{r1} & $\star$ & $\bullet$ & prints \textit{r1} characters from \textit{RSI}\\
        \hline

        HLT & 0xFF & $\star$ & $\star$ & & stops the CPU\\
        \hline  
    \end{tabular}
    \caption{8-bit instruction set}
    \label{tab:8bit_instructions}
\end{table}

\subsection{Program syntax}

To write code, you start by defining what will be stored in memory by declaring 1 byte at a time using \texttt{DB}. In the middle of \texttt{DB} statements, you can use \texttt{DREG} to instantiate a 16-byte space to dump registers. After defining the memory, you can start writing your code using the instruction set table. Note that there is no punctuation and no way to write comments.\\
By following this, a program will starts with 2 bytes that indicates where the first instruction is located. It is then followed by a data space and it is finished by the instructions.



\end{document}
